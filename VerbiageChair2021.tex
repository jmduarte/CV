
\documentclass[11pt]{article}
\usepackage{times}
\usepackage[pazoGreek]{heppennames2}
\usepackage{ptdr-definitions}
\usepackage{geometry}
\begin{document}

Javier Duarte is an experimental high energy physicist, who studies the basic building blocks of matter at the highest energies producible in the laboratory.
The most energetic collider in the world is currently the Large Hadron Collider (LHC) at CERN in Geneva, Switzerland.
Duarte is a member of one of the large experimental collaborations consisting of several thousand scientists called CMS.
Such a large collaboration is needed to build the complex detector, operate it in the extreme conditions of the LHC beam, acquire the data from the collisions, and analyze the data to infer the properties and interactions of the subatomic particles produced in the collisions.
Nonetheless, Duarte is already recognized as an intellectual leader within the collaboration and the field.
The major research question that Duarte seeks to answer is what is the nature of the \emph{new physics} beyond the standard model of particle physics, the current best description of the elementary particles and the fundamental forces.

The direction Duarte pursues is to study the most recently discovered elementary particle, the Higgs boson, to see if its properties deviate from expectations, as this may provide crucial hints as to the nature of the new physics.
His research in CMS currently focuses on searches for and measurements of Higgs bosons decaying to quarks with large transverse momentum ($\pt$).
By studying the production of Higgs bosons at high $\pt$, Duarte's work is sensitive to new physics at very high energy scales.
Within CMS, he led this search for high-$\pt$ Higgs bosons decaying to bottom quark-antiquark pairs ($\bbbar$)~[\textbf{C.1}], which is now submitted to \emph{J. High Energy Phys}.
The search is unique because it probes the highest energy Higgs bosons and uses machine learning (ML) techniques that he developed to better identify its decays.

To enable these physics results, Duarte is a leader in the interdisciplinary subfield of ML for particle physics.
Rather than being explicitly programmed, ML algorithms learn directly from data and simulation to optimally perform a given task such as distinguishing Higgs bosons from background.
Duarte developed novel ML algorithms, based on graph neural networks, to identify the \emph{jet}, or shower of particles, originating from the decay of a Higgs boson~[\textbf{A.I.a.829}, \textbf{A.I.a.883}].
These were shown to be much more effective than other algorithms and will have a large impact on improving the precision of measurements of high-$\pt$ Higgs bosons.

Duarte also leads efforts to develop and implement ML algorithms in dedicated hardware for very fast processing, with a \emph{latency} or delay of less than one microsecond, to improve the real-time LHC event selection, known as the trigger.
As the trigger selects the subset of collision events to be preserved for analysis, while discarding the rest, it is of paramount importance.
For this work, he designed, implemented, and studied fast ML algorithms on field-programmable gate arrays (FPGAs) for particle physics~[\textbf{A.I.a.798}, \textbf{A.I.a.866}, \textbf{A.I.a.874}].
This work builds on a compiler framework that Duarte established previously~[\textbf{A.I.a.647}].
The impact of this work will also be tremendous: by enabling sophisticated ML algorithms to run in the trigger, events that may contain subtle signatures of new physics may be identified and preserved rather than lost forever.

Duarte also made significant contributions to several other CMS publications involving jet-based searches for new physics~[\textbf{A.I.a.822}, \textbf{A.I.a.823}, \textbf{A.I.a.871}, \textbf{A.I.a.881}] through development of the analysis, mentorship of graduate students, and coordination the analysis teams.
Duarte served as the co-convener of the CMS physics analysis subgroup for exotic physics searches with jets in the final state, which has recently produced six publications, with six more in the pipeline.
As a member of the CMS Collaboration, he is also an author of all CMS publications since 2011, many of which benefit from his indirect contributions.

Duarte's growing research group includes two graduate students and two undergraduate students, who have already secured summer resarch funding through the Triton Research and Experiential Learning Scholars (TRELS) program (\$5,000 stipend) and an IRIS-HEP fellowship (\$6,000 stipend).

Exceptionally, Duarte was awarded a Department of Energy (DOE) Early Career Award (ECA) for ``Real-Time Artificial Intelligence for Particle Reconstruction and Higgs Physics'' (\$750,000 over five years) in his first year as junior faculty.
Notably, he is the first ever DOE ECA recipient in high energy physics at UC San Diego.
He also received NSF funding as a Co-PI for the construction of a new AI-centric supercomputer at SDSC with a project entitled ``Category II: Exploring Neural Network Processors for AI in Science and Engineering'' (\$5,000,000 total award) and DOE funding as a Co-PI for ``FAIR Framework for Physics-Inspired Artificial Intelligence in High Energy Physics'' (\$2,250,000 total award).

Finally, Duarte's teaching and service are admirable.
During spring quarter 2020, Duarte taught Physics 2C to 350 students and 98.1\% of the students responding to CAPES recommended him as an instructor.
He also served on the Physics Department Graduate Admissions Committee, spoke at the Young Physicists Program (YPP) for local high school students, peer-reviewed papers for \emph{Eur. Phys. J. C} and \emph{Nucl. Instrum. Methods Phys. Res. A}, and reviewed grant applications for the European Science Foundation and the French National Research Agency.
As part of his commitment to improving diversity, equity, and inclusion in physics, he actively mentors several Latinx students.


\end{document}
