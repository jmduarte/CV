% Also in Overleaf: https://www.overleaf.com/4971656135tgmkrzqprbrp

\documentclass{article}
\usepackage[a5paper, landscape, margin=0in]{geometry}
\usepackage[utf8]{inputenc}
\usepackage{feynmp-auto}
\usepackage{graphicx}
\usepackage{caption}
\usepackage{subcaption}
\usepackage{xspace}

\newcommand{\kappalambda}{\ensuremath{\kappa_\lambda}\xspace}
\newcommand{\kappatop}{\ensuremath{\kappa_\mathrm{t}}\xspace}
\newcommand{\kappabottom}{\ensuremath{\kappa_\mathrm{b}}\xspace}
\newcommand{\kappav}{\ensuremath{\kappa_{\mathrm{V}}\xspace}}
\newcommand{\kappavv}{\ensuremath{\kappa_{2\mathrm{V}}\xspace}}

\begin{document}

% ggF, H, triangle diagram
\begin{figure}
  \centering
  \resizebox{0.8\textwidth}{!}{
    \begin{fmffile}{feyngraph1}
      \begin{fmfgraph*}(170,60)
        \fmfstraight
        \fmfleft{i1,i2}
        \fmfright{o1,m,o2}
        % gluons
        \fmf{gluon}{i1,t1}
        \fmf{gluon}{t2,i2}
        \fmf{phantom,tension=0.4}{t1,o1}
        \fmf{phantom,tension=0.4}{t2,o2}
        \fmffreeze
        % top loop
        \fmf{fermion,tension=1, label=t, label.side=left}{t1,t2,t3,t1}
        \fmf{phantom,tension=1.4}{t3,m}
        \fmffreeze
        % Higgs boson
        \fmf{dashes,tension=1.4,label=H, label.side=left}{t3,h}
        \fmf{fermion,tension=1}{o1,h}
        \fmf{fermion,tension=1}{h,o2}
        \fmf{phantom,tension=1.4}{h,o2}
        \fmf{phantom,tension=1.4}{h,o1}
        % labels
        \fmflabel{g}{i1}
        \fmflabel{g}{i2}
        \fmflabel{b}{o2}
        \fmflabel{$\overline{\mathrm{b}}$}{o1}
        \fmfdot{t3,h}
      \end{fmfgraph*}
    \end{fmffile}
  }
\end{figure}

\clearpage
% ggF, HH, triangle diagram
\begin{figure}
  \centering
  \resizebox{0.8\textwidth}{!}{
    \begin{fmffile}{feyngraph2}
      \begin{fmfgraph*}(80,60)
        \fmfstraight
        \fmfleft{i1}
        \fmfright{o1,o2}
        \fmf{dashes,tension=1.8}{i1,h}
        \fmf{dashes,tension=1}{h,o1}
        \fmf{dashes,tension=1}{h,o2}
        \fmfv{label=$\lambda$,label.angle=-90}{h}
        \fmfdot{h}
        \fmflabel{H}{i1}
        \fmflabel{H}{o1}
        \fmflabel{H}{o2}
      \end{fmfgraph*}
    \end{fmffile}
  }
\end{figure}

\end{document}

